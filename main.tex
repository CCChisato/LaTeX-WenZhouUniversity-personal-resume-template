\documentclass[11pt]{article}

\usepackage{hyperref}
\usepackage{xcolor}
\usepackage{calc}
\usepackage{graphicx}
\usepackage{tikz}
\usepackage{fontspec}
\usepackage{fontawesome5}
\usepackage{titlesec}
\usepackage{enumitem}
\usepackage{fancybox}

\hypersetup{hidelinks}

%%%%%%%%%%%%%%%%%%%%
% 设置
%%%%%%%%%%%%%%%%%%%%

\setlength{\parindent}{0pt}					% 取消全局段落缩进
\pagenumbering{gobble}						% 取消页码显示
\setlist[itemize]{nosep                     % 取消 itemize 的默认间距
    , before={\vspace*{-\parskip}}          % 取消 itemize 和后续段落之间的空白
    , leftmargin=*}		                    % 取消 itemize 的左边距
\setlist[enumerate]{leftmargin=*}	        % 取消 enumerate 的左边距
\renewcommand{\arraystretch}{1.2}           % 设置表格行间距
\linespread{1.25}                           % 设置正文行间距

\titleformat{\section}					    % 将原标题前面的数字取消了
  {\LARGE\bfseries\raggedright} 		      % 字体改为 LARGE,bold,左对齐
  {}{0em}                      			  % 可用于添加全局标题前缀
  {}                           			  % 可用于添加代码
  [{\color{secondary_color}\titlerule}]     % 标题下方加一条线
\titlespacing*{\section}{0cm}{*1.2}{*1.2}	% 标题左边留白,上方,下方

\usepackage[
	a4paper,
	left=1.2cm,
	right=1.2cm,
	top=1.5cm,
	bottom=1cm,
	nohead
]{geometry}                                 % 页面边距设置

% 字体设置
\setmainfont[
    Path=fonts/,
    Extension=.otf,
    BoldFont=*-Bold,
]{NotoSerifSC}

% 自定义颜色(参考 https://github.com/seumxc/SEU-Logo)
\definecolor{primary_color}{RGB}{30, 58, 138}     % 温州大学主蓝 #1E3A8A
\definecolor{secondary_color}{RGB}{40, 70, 150}   % 稍浅蓝(用于下划线等)

\newlength{\iconwidth}
\setlength{\iconwidth}{1.5em}                   % 设置 section 标题部分图标占用的宽度

%%%%%%%%%%%%%%%%%%%%
% 文章内容
%%%%%%%%%%%%%%%%%%%%

% 学院
\newcommand{\school}{学院名称 | School Name}

% 联系方式
\newcommand{\contact}{
    % 根据个人喜好选择字号
    % \small                % 小
    \footnotesize           % 更小
    % \scriptsize           % 再小一号
    \textcolor{white}{
        % 邮箱
        \faEnvelope \quad \href{mailto:your.email@example.com}{your.email@example.com}
        \hspace{4em}
        % 手机号
        \faPhone \quad  XXX-XXXX-XXXX
        % 别的联系方式,如微信、GitHub等
        \hspace{4em}
        \faGithub \quad \href{https://github.com/yourusername}{github.com/yourusername}
    }
}

\begin{document}

    %%%%%%%%%%%%%%%%%%%%
    % 页眉、页脚和背景(如果有多页简历,请把页眉页脚和背景复制粘贴到第二页的内容之前)
    %%%%%%%%%%%%%%%%%%%%

    % 页眉:校标组合+学院名
    \begin{tikzpicture}[remember picture, overlay]
        \node[anchor=north, inner sep=0pt](header) at (current page.north){
            \includegraphics[width=\paperwidth]{images/head_wzu.png}
        };
        \node[anchor=west](school_logo) at (header.west){
            \hspace{0.5cm}
            \includegraphics[width=0.08\textwidth]{images/wzu_logo.png}
        };
        \node[anchor=east](school_name) at(header.east){
            \textcolor{white}{\textbf{\school}}
            \hspace{0.5cm}
        };
    \end{tikzpicture}
    \vspace{-3.5em}


    % 页脚,联系方式
    \begin{tikzpicture}[remember picture, overlay]
        \node[anchor=south, inner sep=0pt](footer) at (current page.south){
            \includegraphics[width=\paperwidth]{images/foot_wzu.png}
        };
        \node[anchor=center] at(footer.center){
            \textcolor{white}{\contact} \\
            \footnotesize \textcolor{white}{\qquad 本简历使用 \LaTeX\ 排版,源码托管于 GitHub}
        };
    \end{tikzpicture}

    % 背景
    \begin{tikzpicture}[remember picture, overlay]
        \node[opacity=0.05] at(current page.center){
            \includegraphics[width=0.7\paperwidth, keepaspectratio]{images/wzu_logo.png}
        };
    \end{tikzpicture}

    %%%%%%%%%%%%%%%%%%%%
    % 简历正文
    %%%%%%%%%%%%%%%%%%%%

    \begin{minipage}[t]{0.78\textwidth}
        % 个人信息
        % \faGraduationCap这类\fa开头的都是font awesome里的logo,想换成其他logo的话,可以看一下附带的fontawsome.pdf,自行替换。
        \section[个人信息]{\makebox[\iconwidth][c]{\color{primary_color}{\faAddressCard}}\quad 个人信息}
        \begin{minipage}[t]{\textwidth}
            \textbf{姓\quad 名}:姓名 \\
            \textbf{邮\quad 箱}:your.email@example.com \\
            \textbf{电\quad 话}:XXX-XXXX-XXXX \\
            \textbf{GitHub}:\href{https://github.com/yourusername}{github.com/yourusername} \\
        \end{minipage}

        % 教育背景
        \begin{minipage}[t]{\textwidth}
        \section[教育背景]{\makebox[\iconwidth][c]{\color{primary_color}{\faGraduationCap}}\quad 教育背景}
        
        {\large \textbf{大学名称}},学历 \hfill 20XX年X月--20XX年X月(预计)  
        \begin{itemize}
            \item 学院名称,专业名称
            \item \textbf{主修课程}:课程1、课程2、课程3、课程4、课程5、课程6等。
        \end{itemize}
        
        \vspace{1.2em}
        \end{minipage}
    \end{minipage}
    \hfill
    % 右半边,照片,比例占行宽20%
    \begin{minipage}[t]{0.2\textwidth}
        \vspace{2em} % 照片上侧内容
        \setlength{\fboxsep}{0pt}
        \doublebox{\includegraphics[width=\linewidth]{images/person.png}}
    \end{minipage}

    \begin{minipage}[t]{\textwidth}
    % 项目与技术实践
    \section[项目与技术实践]{\makebox[\iconwidth][c]{\color{primary_color}{\faCode}}\quad 项目与技术实践}

        {\large \textbf{项目名称1}} \hfill github.com/yourusername/project1
        \item \textbf{角色}:角色描述
        \item \textbf{技术栈}:技术1 + 技术2 + 技术3 + 技术4
        \item \textbf{功能}:功能描述
        \item \textbf{亮点}:
        \begin{itemize}
            \item 项目亮点描述
        \end{itemize}
    
    \vspace{0.5em}
    
    {\large \textbf{项目名称2}} \hfill 项目来源
    \item \textbf{角色}:角色描述
        \item \textbf{技术栈}:技术1 + 技术2 + 技术3 + 技术4 + 技术5
        \item \textbf{功能}:功能描述
        \item \textbf{亮点}:
        \begin{itemize}
            \item 亮点描述1
            \item 亮点描述2
            \item 亮点描述3
        \end{itemize}
    
    \vspace{0.5em}
    \end{minipage}

    \begin{minipage}[t]{\textwidth}
    % 项目经历\科研经历\项目与教学(标题请根据需要修改)

    \section[个人成长]{\makebox[\iconwidth][c]{\color{primary_color}{\faBookOpen}}\quad 个人成长与技能}

    个人经历概述,包括实践经历、团队协作能力等。
    
    \item \textbf{技能类别1}:技能描述,包括掌握的技术、工具等。
    
    \item \textbf{技能类别2}:技能描述,包括项目经验、技术深度等。
    
    \item \textbf{技能类别3}:技能描述,包括使用的工具、软件等。
    
    \item \textbf{技能类别4}:技能描述,包括学习能力、信息获取能力等。
    \end{minipage}
\end{document}